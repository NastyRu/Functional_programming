\documentclass[a4paper,14pt]{extreport} % формат документа

\usepackage{amsmath}
\usepackage{cmap} % поиск в ПДФ
\usepackage[T2A]{fontenc} % кодировка
\usepackage[utf8]{inputenc} % кодировка исходного текста
\usepackage[english,russian]{babel} % локализация и переносы
\usepackage[left = 2cm, right = 1cm, top = 2cm, bottom = 2 cm]{geometry} % поля
\usepackage{listings}
\usepackage{graphicx} % для вставки рисунков
\usepackage{amsmath}
\usepackage{float}
\usepackage{multirow}
\graphicspath{{pictures/}}
\DeclareGraphicsExtensions{.pdf,.png,.jpg}
\newcommand{\anonsection}[1]{\section*{#1}\addcontentsline{toc}{section}{#1}}

\lstset{ %
	language=Lisp,                % Язык программирования 
	numbers=left,                   % С какой стороны нумеровать          
	frame=single,                    % Добавить рамку
}

\begin{document}
\begin{titlepage}

    \begin{table}[H]
        \centering
        \footnotesize
        \begin{tabular}{cc}
            \multirow{8}{*}{\includegraphics[scale=0.35]{bmstu.jpg}}
            & \\
            & \\
            & \textbf{Министерство науки и высшего образования Российской Федерации} \\
            & \textbf{Федеральное государственное бюджетное образовательное учреждение} \\
            & \textbf{высшего образования} \\
            & \textbf{<<Московский государственный технический} \\
            & \textbf{университет имени Н.Э. Баумана>>} \\
            & \textbf{(МГТУ им. Н.Э. Баумана)} \\
        \end{tabular}
    \end{table}

    \vspace{-2.5cm}

    \begin{flushleft}
        \rule[-1cm]{\textwidth}{3pt}
        \rule{\textwidth}{1pt}
    \end{flushleft}

    \begin{flushleft}
        \small
        ФАКУЛЬТЕТ
        \underline{<<Информатика и системы управления>>\ \ \ \ \ \ \ 
        \ \ \ \ \ \ \ \ \ \ \ \ \ \ \ \ \ \ \ \ \ \ \ \ \ \ \ \ \ \ \ 
    \ \ \ \ \ \ \ \ \ \ \ \ \ \ \ } \\
        КАФЕДРА
        \underline{<<Программное обеспечение ЭВМ и
        информационные технологии>>
        \ \ \ \ \ \ \ \ \ \ \ \ \ \ \ \ \ \ \ \ }
    \end{flushleft}

    \vspace{4cm}

    \begin{center}
        \textbf{Лабораторная работа № 10} \\
        \vspace{0.5cm}
        \textbf{} \\
    \end{center}

    \vspace{4cm}

    \begin{flushleft}
        \begin{tabular}{ll}
            \textbf{Дисциплина} & Функциональное и логическое программирование \\
            \textbf{Студент} & Сиденко А.Г. \\
            \textbf{Группа} & ИУ7-63Б \\
            \textbf{Преподаватель} & Толпинская Н.Б., Строганов Ю.В.  \\
        \end{tabular}
    \end{flushleft}

    \vspace{4cm}

   \begin{center}
        Москва, 2020 г.
    \end{center}

\end{titlepage}

\begin{enumerate}
\item \textbf{Пусть list-of-list список, состоящий из списков. Написать функцию, которая вычисляет сумму длин всех элементов list-of-list.}

На вход список. 

Пока список ненулевой прибавляем к длине головы функцию от хвоста (вернет длину хвоста). Когда дойдет до конца функция вернет 0. 

\begin{lstlisting}
(defun len_lists(lst)
  (if lst
    (+ (length (car lst)) (len_lists (cdr lst)))
    0
  )
)
\end{lstlisting}

\textbf{Примеры:}

\begin{lstlisting}
(len_lists '((4 44 3 2 ) (2 3 4) (6)))
\end{lstlisting}

Результат: 8

\item \textbf{Написать рекурсивную версию (с именем reс-add) вычисления суммы чисел заданного списка.}

На вход список. 

Идея та же, что и в предыдущем номере, только прибавляем не длину, а само значение. 

\begin{lstlisting}
(defun rec-add(lst)
  (if lst
    (+ (car lst) (rec-add (cdr lst)))
    0
  )
)
\end{lstlisting}

\textbf{Примеры: }

\begin{lstlisting}
(rec-add '(1 2 3 4 5))
\end{lstlisting}

Результат: 15

\item \textbf{Написать рекурсивную версию с именем recnth функции nth.}

На вход список и число(позиция). 

Функция recnth, вызывает рекурсивную функцию recnth$\_$next с параметрами список и число (длина списка, который должен остаться), которая как раз в свою очередь и находит необходимую нам позицию в списке. 

\begin{lstlisting}
(defun recnth_next(lst k)
  (cond
    ((= k (length lst)) (car lst))
    (t (recnth_next (cdr lst) k))
  )
)

(defun recnth(lst k)
  (cond
    ((> k (length lst)) Nil)
    ((< k 1) Nil)
    (t (recnth_next lst (+ 1 (- (length lst) k))))
  )
)
\end{lstlisting}

\textbf{Примеры:}

\begin{lstlisting}
(recnth '(1 2 3 4 5) 6)
\end{lstlisting}

Результат: Nil

\begin{lstlisting}
(recnth '(1 (2 1) (3 4 5) 4 5) 3)
\end{lstlisting}

Результат: (3 4 5)

\item \textbf{Написать рекурсивную функцию alloddr, которая возвращает t когда все элементы списка нечетные.}

На вход список. 

Пока список ненулевой применяем and к проверки на четность головы и функции от хвоста (вернет t или Nil хвоста). Когда дойдет до конца функция вернет t. 

\begin{lstlisting}
(defun alloddr(lst)
  (if lst
    (and (oddp (car lst)) (alloddr (cdr lst)))
    t
  )
)
\end{lstlisting}

\textbf{Примеры:}

\begin{lstlisting}
(alloddr '(1 3 5 7 9))
\end{lstlisting}

Результат: T

\begin{lstlisting}
(alloddr '(1 2 3 4 5))
\end{lstlisting}

Результат: Nil

\item \textbf{Написать рекурсивную функцию, относящуюся к хвостовой рекурсии с одним тестом завершения, которая возвращает последний элемент списка-аргумента.}

Пока второй элемент не Nil идем дальше, иначе возвращаем голову
 
\begin{lstlisting}
(defun last_elem (lst)
  (if (NULL (cadr lst)) (car lst) 
    (last_elem (cdr lst))
  ) 
)
\end{lstlisting}

\textbf{Примеры:}

\begin{lstlisting}
(last_elem '((1) (2 4 5) (3 4 5 (6 7)) (4 (-3) 8)))
\end{lstlisting}

Результат: (4 (-3) 8)

\item \textbf{Написать рекурсивную функцию, относящуюся к дополняемой рекурсии с одним тестом завершения, которая вычисляет сумму всех чисел от 0 до n-ого аргумента функции. Вариант: 1) от n-аргумента функции до последнего; 2) от n-аргумента функции до m-аргумента с шагом d. }

Дополняемая рекурсия, считающая сумму всех чисел от 0 до n-ого аргумента. Используем дополнительную функцию, так как необходимо преобразовать индекс к длине хвоста. Далее просто прибавляются все числа, пока длина хвоста длиннее аргумента. 

На вход список и число, порядок аргумента. 

\begin{lstlisting}
(defun sum_next(lst n)
  (cond
    ((= n (length lst)) (car lst))
    (t (+ (car lst) (sum_next (cdr lst) n)))
  )
)

(defun sumN(lst n)
  (sum_next lst (+ 1 (- (length lst) n)))
)
\end{lstlisting}

\textbf{Примеры:}

\begin{lstlisting}
(sumN '(1 2 3 4 5 6 7) 5)
\end{lstlisting}

Результат: 15

Дополняемая рекурсия, считающая сумму всех чисел от n-ого аргумента до конца. Используем дополнительную функцию, так как необходимо преобразовать индекс к длине хвоста. Далее просто прибавляются все числа, пропускаются те элементы, где длина хвоста длиннее, прибавляем голову, где длина хвоста равна аргументу и далее до конца (длина равна 1). 

На вход список и число, порядок аргумента. 

\begin{lstlisting}
(defun sum_nextlast(lst n)
  (cond
    ((= 1 (length lst)) (car lst))
    ((>= n (length lst)) (+ (car lst) 
    	(sum_nextlast (cdr lst) n))
    )
    (t (sum_nextlast (cdr lst) n))
  )
)

(defun sumlastN(lst n)
  (sum_nextlast lst (+ 1 (- (length lst) n)))
)
\end{lstlisting}

\textbf{Примеры:}

\begin{lstlisting}
(sumlastN '(1 2 3 4 5 6 7) 5)
\end{lstlisting}

Результат: 18

Дополняемая рекурсия, считающая сумму всех чисел от n-ого аргумента до m-го с шагом d. Используем дополнительную функцию, так как необходимо преобразовать индекс к длине хвоста. Далее просто прибавляются все числа, пропускаются те элементы, где длина хвоста длиннее n, прибавляем голову, где длина хвоста равна n  и далее до m, причем прибавляются только числа позиция которых кратна шагу. 

На вход список и 3 числа, порядок аргументов и шаг. 

\begin{lstlisting}
(defun sum_nextNMD(lst n m d)
  (let ((len (length lst)))
    (cond
      ((= m len)
        (if (= 0 (rem (- n m) 3))
          (car lst)
          0
        )
      )
      ((and (>= n len) (= 0 (rem (- n len) 3)))
        (+ (car lst) (sum_nextNMD (cdr lst) n m d))
      )
      (t
        (sum_nextNMD (cdr lst) n m d)
      )
    )
  )
)

(defun sumNMD(lst n m d)
  (let ((len (length lst)))
    (sum_nextNMD lst (+ 1 (- len n)) (+ 1 (- len m)) d)
  )
)
\end{lstlisting}

\textbf{Примеры:}

\begin{lstlisting}
(sumNMD '(1 2 3 4 5 6 7 8 9 10) 2 9 3)
\end{lstlisting}

Результат: 15

\item \textbf{Написать рекурсивную функцию, которая возвращает последнее нечетное число из числового списка, возможно создавая некоторые вспомогательные функции.}

Рекурсивная функция: список пустой, возвращает Nil; если число четное -- возвращается значение рекурсии, если нечетное -- проверяется, что вернула рекурсия, если вернула не Nil(было найдено нечетное число), то возвращается ранее найденное число(оно будет в списке позже); если вернула Nil то возвращается само нечетное число. 

На вход список. 

\begin{lstlisting}
(defun last_odd (lst)
  (cond ((null lst) Nil)
        ((oddp (car lst))
          (let ((rec (last_odd (cdr lst))))
            (if (null rec) (car lst) rec)
          )
        )
        (t (last_odd (cdr lst)))
  )
)
\end{lstlisting}

\textbf{Примеры:}

\begin{lstlisting}
(last_odd '(1 2 3 4 5 6 7))
\end{lstlisting}

Результат: 7

\item \textbf{Используя cons-дополняемую рекурсию с одним тестом завершения, написать функцию которая получает как аргумент список чисел, а возвращает список квадратов этих чисел в том же порядке.}

Пока список не Nil, объединяем все числа в списке, возведенные в квадрат. 

На вход список. 

\begin{lstlisting}
(defun square (lst)
  (if lst
    (cons
      (* (car lst) (car lst))
      (square (cdr lst))
    )
  )
)
\end{lstlisting}

\textbf{Примеры:}

\begin{lstlisting}
(square '(1 -2 3 4 5))
\end{lstlisting}

Результат: (1 4 9 16 25)

\item \textbf{Написать функцию с именем select-odd, которая из заданного списка выбирает все нечетные числа. (Вариант 1: select-even, вариант 2: вычисляет сумму всех нечетных чисел(sum-all-odd) или сумму всех четных чисел (sum-all-even) из заданного списка. )}

На вход список. 

Пока список не Nil (каждый раз функция применяется для хвоста списка), ищутся все нечетные числа и создается список. 

\begin{lstlisting}
(defun select_odd (lst)
  (if lst
    (append
      (if (oddp (car lst))
      	(list (car lst)) Nil
      )
      (select_odd (cdr lst))
    )
  )
)
\end{lstlisting}

\textbf{Примеры:}

\begin{lstlisting}
(select_odd '(1 2 3 4 5 6 7))
\end{lstlisting}

Результат: (1 3 5 7)

Пока список не Nil (каждый раз функция применяется для хвоста списка), ищутся все четные числа и создается список. 

\begin{lstlisting}
(defun select_even (lst)
  (if lst
    (append
      (if (evenp (car lst))
      	(list (car lst)) Nil
      )
      (select_even (cdr lst))
    )
  )
)
\end{lstlisting}

\textbf{Примеры:}

\begin{lstlisting}
(select_even '(1 2 3 4 5 6 7))
\end{lstlisting}

Результат: (2 4 6)

Пока список не Nil (каждый раз функция применяется для хвоста списка), ищутся все нечетные числа и считается их сумма. 

\begin{lstlisting}
(defun sum_all_odd (lst)
  (cond ((null lst) 0)
        ((oddp (car lst))
          (+ (car lst) (sum_all_odd (cdr lst)))
        )
        (t (sum_all_odd (cdr lst)))
  )
)
\end{lstlisting}

\textbf{Примеры:}

\begin{lstlisting}
(sum_all_odd '(1 2 3 4 5 6 7))
\end{lstlisting}

Результат: 16

Пока список не Nil (каждый раз функция применяется для хвоста списка), ищутся все четные числа и считается их сумма. 

\begin{lstlisting}
(defun sum_all_even (lst)
  (cond ((null lst) 0)
        ((evenp (car lst))
          (+ (car lst) (sum_all_even (cdr lst)))
        )
        (t (sum_all_even (cdr lst)))
  )
)
\end{lstlisting}

\textbf{Примеры:}

\begin{lstlisting}
(sum_all_even '(1 2 3 4 5 6 7))
\end{lstlisting}

Результат: 12

\end{enumerate}

\textbf{Теоретические вопросы:}

\begin{enumerate}
\item Способы организации повторных вычислений в Lisp. 

Использование функционалов или рекурсии. 

\item Что такое рекурсия? Классификация рекурсивных функций в Lisp. 

Рекурсия -- это ссылка на определяемый объект во время его определения. 

\begin{enumerate}
\item Простая рекурсия -- один рекурсивный вызов в теле. 
\item Рекурсия первого порядка -- рекурсивный вызов встречается несколько раз. 
\item Взаимная рекурсия -- используется несколько функций, рекурсивно вызывающих друг друга.
\end{enumerate}

\item Различные способы организации рекурсивных функций и порядок их реализации. 

\begin{enumerate}
\item Хвостовая рекурсия -- результат формируется не на выходе из рекурсии, а на входе в рекурсию, все действия выполняя до ухода на следующий шаг рекурсии. 
\item Рекурсия по нескольким параметрам
\item Дополняемая рекурсия -- при обращении к рекурсивной функции используется дополнительная функция не в аргументе вызова, а вне его
\item Множественная рекурсия -- на одной ветке происходит сразу несколько рекурсивных вызовов.
\end{enumerate}

\item Способы повышения эффективности реализации рекурсии. 

Использование хвостовой рекурсии. Если условий выхода несколько, то надо думать о порядке их следования. Некачественный выход из рекурсии может привести к переполнению памяти из-за "лишних" рекурсивных вызовов. 

Преобразование не хвостовой рекурсии в хвостовую, возможно путем использования дополнительных параметров. В этом случае необходимо использовать функцию-оболочку для запуска рекурсивной функции с начальными значениями дополнительных параметров. 

\end{enumerate}
\end{document}