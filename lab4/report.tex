\documentclass[a4paper,14pt]{extreport} % формат документа

\usepackage{amsmath}
\usepackage{cmap} % поиск в ПДФ
\usepackage[T2A]{fontenc} % кодировка
\usepackage[utf8]{inputenc} % кодировка исходного текста
\usepackage[english,russian]{babel} % локализация и переносы
\usepackage[left = 2cm, right = 1cm, top = 2cm, bottom = 2 cm]{geometry} % поля
\usepackage{listings}
\usepackage{graphicx} % для вставки рисунков
\usepackage{amsmath}
\usepackage{float}
\usepackage{multirow}
\graphicspath{{pictures/}}
\DeclareGraphicsExtensions{.pdf,.png,.jpg}
\newcommand{\anonsection}[1]{\section*{#1}\addcontentsline{toc}{section}{#1}}

\lstset{ %
	language=Lisp,                % Язык программирования 
	numbers=left,                   % С какой стороны нумеровать          
	frame=single,                    % Добавить рамку
}

\begin{document}
\begin{titlepage}

    \begin{table}[H]
        \centering
        \footnotesize
        \begin{tabular}{cc}
            \multirow{8}{*}{\includegraphics[scale=0.35]{bmstu.jpg}}
            & \\
            & \\
            & \textbf{Министерство науки и высшего образования Российской Федерации} \\
            & \textbf{Федеральное государственное бюджетное образовательное учреждение} \\
            & \textbf{высшего образования} \\
            & \textbf{<<Московский государственный технический} \\
            & \textbf{университет имени Н.Э. Баумана>>} \\
            & \textbf{(МГТУ им. Н.Э. Баумана)} \\
        \end{tabular}
    \end{table}

    \vspace{-2.5cm}

    \begin{flushleft}
        \rule[-1cm]{\textwidth}{3pt}
        \rule{\textwidth}{1pt}
    \end{flushleft}

    \begin{flushleft}
        \small
        ФАКУЛЬТЕТ
        \underline{<<Информатика и системы управления>>\ \ \ \ \ \ \ 
        \ \ \ \ \ \ \ \ \ \ \ \ \ \ \ \ \ \ \ \ \ \ \ \ \ \ \ \ \ \ \ 
    \ \ \ \ \ \ \ \ \ \ \ \ \ \ \ } \\
        КАФЕДРА
        \underline{<<Программное обеспечение ЭВМ и
        информационные технологии>>
        \ \ \ \ \ \ \ \ \ \ \ \ \ \ \ \ \ \ \ \ }
    \end{flushleft}

    \vspace{2cm}

    \begin{center}
        \textbf{Лабораторная работа № 6} \\
        \vspace{0.5cm}
        \textbf{Рекурсивные функции.  } \\
    \end{center}

    \vspace{4cm}

    \begin{flushleft}
        \begin{tabular}{ll}
            \textbf{Дисциплина} & Функциональное и логическое программирование \\
            \textbf{Студент} & Сиденко А.Г. \\
            \textbf{Группа} & ИУ7-63Б \\
            \textbf{Преподаватель} & Толпинская Н.Б.  \\
        \end{tabular}
    \end{flushleft}

    \vspace{4cm}

   \begin{center}
        Москва, 2020 г.
    \end{center}

\end{titlepage}

\begin{enumerate}
\item \textbf{Написать функцию, которая переводит температуру в системе Фаренгейта в температуру по Цельсию}

\begin{lstlisting}
(defun f-to-c (temp)
  (* (/ 5 9) (- temp 32.0)) )
  
(write (f-to-c 451)) 
\end{lstlisting}

Результат: 232.77779

\item \textbf{Что получится при вычисления каждого из выражений?}

\begin{lstlisting}
(list 'cons t NIL)
; (CONS T NIL)
\end{lstlisting}
Создает 3 списковых ячейки, возвращает список

\begin{lstlisting}
(eval (eval (list 'cons t NIL)))
; Undefined function:  T
\end{lstlisting}
EVAL -- позволяет вычислять значения выражений, представленных в виде списков. 

\begin{lstlisting}
(apply #cons ''(t NIL))
; illegal complex number format: #CONS
\end{lstlisting}
Знак решётки обозначает начало сложной синтаксической конструкции (числа в различных системах счисления).

\begin{lstlisting}
(list 'eval NIL)
; (EVAL NIL)
\end{lstlisting}

\begin{lstlisting}
(eval (list 'cons t NIL))
; (T)
\end{lstlisting}
EVAL аннулирует кавычку

\begin{lstlisting}
(eval NIL)
; NIL
\end{lstlisting}

\begin{lstlisting}
(eval (list 'eval NIL))
; NIL
\end{lstlisting}

\item \textbf{Написать функцию, вычисляющую катет по заданной гипотенузе и другому катету прямоугольного треугольника, и составить диаграмму ее вычисления.}

\begin{lstlisting}
(defun cathet (cat hyp)
  (sqrt (- (* hyp hyp) (* cat cat))) )

(write (cathet 3 5))
\end{lstlisting}

Результат: 4.0

\hfill

\hfill

\hfill

\hfill

\hfill

\hfill

\hfill

\hfill

\hfill


\item \textbf{Написать функцию, вычисляющую площадь трапеции по ее основаниям и высоте, и составить диаграмму ее вычисления.}

\begin{lstlisting}
(defun square_trap (a b h)
  (* (/ (+ a b) 2) h) 

(write (square_trap 2 2 2))
\end{lstlisting}

Результат: 4

\end{enumerate}
\end{document}