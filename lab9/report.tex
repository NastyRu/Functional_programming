\documentclass[a4paper,14pt]{extreport} % формат документа

\usepackage{amsmath}
\usepackage{cmap} % поиск в ПДФ
\usepackage[T2A]{fontenc} % кодировка
\usepackage[utf8]{inputenc} % кодировка исходного текста
\usepackage[english,russian]{babel} % локализация и переносы
\usepackage[left = 2cm, right = 1cm, top = 2cm, bottom = 2 cm]{geometry} % поля
\usepackage{listings}
\usepackage{graphicx} % для вставки рисунков
\usepackage{amsmath}
\usepackage{float}
\usepackage{multirow}
\graphicspath{{pictures/}}
\DeclareGraphicsExtensions{.pdf,.png,.jpg}
\newcommand{\anonsection}[1]{\section*{#1}\addcontentsline{toc}{section}{#1}}

\lstset{ %
	language=Lisp,                % Язык программирования 
	numbers=left,                   % С какой стороны нумеровать          
	frame=single,                    % Добавить рамку
}

\begin{document}
\begin{titlepage}

    \begin{table}[H]
        \centering
        \footnotesize
        \begin{tabular}{cc}
            \multirow{8}{*}{\includegraphics[scale=0.35]{bmstu.jpg}}
            & \\
            & \\
            & \textbf{Министерство науки и высшего образования Российской Федерации} \\
            & \textbf{Федеральное государственное бюджетное образовательное учреждение} \\
            & \textbf{высшего образования} \\
            & \textbf{<<Московский государственный технический} \\
            & \textbf{университет имени Н.Э. Баумана>>} \\
            & \textbf{(МГТУ им. Н.Э. Баумана)} \\
        \end{tabular}
    \end{table}

    \vspace{-2.5cm}

    \begin{flushleft}
        \rule[-1cm]{\textwidth}{3pt}
        \rule{\textwidth}{1pt}
    \end{flushleft}

    \begin{flushleft}
        \small
        ФАКУЛЬТЕТ
        \underline{<<Информатика и системы управления>>\ \ \ \ \ \ \ 
        \ \ \ \ \ \ \ \ \ \ \ \ \ \ \ \ \ \ \ \ \ \ \ \ \ \ \ \ \ \ \ 
    \ \ \ \ \ \ \ \ \ \ \ \ \ \ \ } \\
        КАФЕДРА
        \underline{<<Программное обеспечение ЭВМ и
        информационные технологии>>
        \ \ \ \ \ \ \ \ \ \ \ \ \ \ \ \ \ \ \ \ }
    \end{flushleft}

    \vspace{4cm}

    \begin{center}
        \textbf{Лабораторная работа № 9} \\
        \vspace{0.5cm}
        \textbf{} \\
    \end{center}

    \vspace{4cm}

    \begin{flushleft}
        \begin{tabular}{ll}
            \textbf{Дисциплина} & Функциональное и логическое программирование \\
            \textbf{Студент} & Сиденко А.Г. \\
            \textbf{Группа} & ИУ7-63Б \\
            \textbf{Преподаватель} & Толпинская Н.Б., Строганов Ю.В.  \\
        \end{tabular}
    \end{flushleft}

    \vspace{4cm}

   \begin{center}
        Москва, 2020 г.
    \end{center}

\end{titlepage}

\begin{enumerate}
\item \textbf{Написать предикат set-equal, который возвращает t, если два его множество-аргумента содержат одни и те же элементы, порядок которых не имеет значения.}

На вход 2 списка, проверяется что каждый список является подмножеством другого. 

subsetp -- является предикатом, который возвращает T, если каждый элемент списка list1 встречается в («равен» некоторому элементу в) списке list2, иначе Nil.

\begin{lstlisting}
(defun set-equal (lst1 lst2)
  (and (subsetp lst1 lst2) (subsetp lst2 lst1)) 
)
\end{lstlisting}

\textbf{Примеры:}

\begin{lstlisting}
(set-equal '(1 4 2 3) '(2 3 1 4))  
\end{lstlisting}

Результат: T

\begin{lstlisting}
(set-equal '(1 4 3) '(2 3 1 4))  
\end{lstlisting}

Результат: Nil

\item \textbf{Напишите необходимые функции, которые обрабатывают таблицу из точечных пар:
(страна. столица), и возвращают по стране - столицу, а по столице - страну.}

Сначала необходимо создать точечные пары. 

\begin{lstlisting}
(setq countries '(Russia France Germany USA))
(setq cities '(Moscow Paris Berlin Vashington))

(defun c_point (f z)
  (cons f z) 
)

(setq newPoints (mapcar #'c_point countries cities))
\end{lstlisting}

\textbf{Результат: }

((RUSSIA . MOSCOW) (FRANCE . PARIS) (GERMANY . BERLIN) (USA . VASHINGTON))

\textbf{Рекурсивный поиск} страны по столице и наоборот. 

На вход подается страна/столица и список пар. На выходе при успешном поиске возвращается соответствующее значение или Nil, если значение не найдено. 

\begin{lstlisting}
(defun found_country (city lst)
  (cond ((null lst) nil)
        ((eql city (cdr (car lst))) (caar lst))
        (t (found_country city (cdr lst)))
  )
)

(defun found_city (country lst)
  (cond ((null lst) nil)
        ((eql country (car (car lst))) (cdr (car lst)))
        (t (found_city country (cdr lst)))
  )
)
\end{lstlisting}

\textbf{Примеры: }
\begin{lstlisting}
(found_country 'Moscow newPoints)
\end{lstlisting}

Результат: RUSSIA

\begin{lstlisting}
(found_city 'h newPoints)
\end{lstlisting}

Результат: NIL

\textbf{Поиск} с использование функционалов. 

На вход подается страна/столица и список пар. Определяется функция found, для поиска страны/города. Данная функция будет применяться каскадным образом (к первым двум, затем к результату и следующему и так далее). Следовательно, первый раз первый аргумент будет точечной парой, в последующие атомом, делаем проверку и в зависимости от нее, либо проверяем на совпадение одного из значений точечной пары с аргументом (только в первый раз), либо нет. Второй аргумент проверяется всегда. 

\begin{lstlisting}
(defun found_country_func (city lst)
  (defun found (lst1 lst2)
    (if (consp lst1)
      (or (if (eql city (cdr lst1)) (car lst1) Nil)
          (if (eql city (cdr lst2)) (car lst2) Nil) )
      (or lst1 (if (eql city (cdr lst2)) (car lst2) Nil) )
    )
  )
  (reduce #'found newPoints)
)

(defun found_city_func (country lst)
  (defun found (lst1 lst2)
    (if (consp lst1)
      (or (if (eql country (car lst1)) (cdr lst1) Nil)
          (if (eql country (car lst2)) (cdr lst2) Nil) )
      (or lst1 (if (eql country (car lst2))(cdr lst2) Nil))
    )
  )
  (reduce #'found newPoints)
)
\end{lstlisting}

\textbf{Примеры: }
\begin{lstlisting}
(found_country 'Moscow newPoints)
\end{lstlisting}

Результат: RUSSIA

\begin{lstlisting}
(found_city 'Moscow newPoints)
\end{lstlisting}

Результат: NIL

\item \textbf{Напишите функцию, которая умножает на заданное число-аргумент все числа
из заданного списка-аргумента, когда: a) все элементы списка --- числа, б) элементы списка -- любые объекты.}

На вход функции список и число, на которое умножать. 

\textbf{С использованием функционалов}

Используется функционал mapcar, $\lambda$-функция (умножение элемента на аргумент, во 2-ой функции сначала проверка на число) применяется к первым элементам списков, затем ко вторым и т.д., и результаты применения собираются в результирующий список.

\begin{lstlisting}
(defun multiplication_numbers (lst k)
  (mapcar #'(lambda (x) (* x k)) lst)
)

(defun multiplication_all (lst k)
  (mapcar #'(lambda (x) (if (numberp x) (* x k) x)) lst)
)
\end{lstlisting}

\textbf{Примеры:}

\begin{lstlisting}
(multiplication_numbers '(1 2 3) 10)
\end{lstlisting}

Результат: (10 20 30)

\begin{lstlisting}
(multiplication_all '(1 2 a 3) 10)
\end{lstlisting}

Результат: (10 20 A 30)

\textbf{С использованием рекурсии}

Пока список не Nil (каждый раз функция применяется для хвоста списка), с использованием функции cons создается список из обновленных значений. 

\begin{lstlisting}
(defun mul_numbers_rec (lst k)
  (if lst
    (cons (* (car lst) k) (mul_numbers_rec (cdr lst) k))
  )
)

(defun mul_all_rec (lst k)
  (if lst
    (cons
      (if (numberp (car lst)) (* (car lst) k) (car lst))
      (mul_all_rec (cdr lst) k)
    )
  )
)
\end{lstlisting}

\textbf{Примеры:}

\begin{lstlisting}
(mul_numbers_rec '(1 2 3) 10)
\end{lstlisting}

Результат: (10 20 30)

\begin{lstlisting}
(mul_all_rec '(1 2 a 3) 10)
\end{lstlisting}

Результат: (10 20 A 30)

\item \textbf{Напишите функцию, которая уменьшает на 10 все числа из списка аргумента этой функции.}

На вход функции список. 

\textbf{С использованием функционалов}

Используется функционал mapcar, $\lambda$-функция (если число, вычитание 10 из аргумента, иначе оставляем тоже значение) применяется к первым элементам списков, затем ко вторым и т.д., и результаты применения собираются в результирующий список.

\begin{lstlisting}
(defun minus_ten (lst)
  (mapcar #'(lambda (x) (if (numberp x) (- x 10) x)) lst)
)
\end{lstlisting}

\textbf{Примеры:}

\begin{lstlisting}
(minus_ten '(1 2 a 3))
\end{lstlisting}

Результат: (-9 -8 A -7)

\textbf{С использованием рекурсии}

Пока список не Nil (каждый раз функция применяется для хвоста списка), с использованием функции cons создается список из обновленных значений. 

\begin{lstlisting}
(defun minus_ten_rec (lst)
  (if lst
    (cons
      (if (numberp (car lst)) (- (car lst) 10) (car lst))
      (minus_ten_rec (cdr lst))
    )
  )
)
\end{lstlisting}

\textbf{Примеры:}

\begin{lstlisting}
(minus_ten_rec '(1 2 a 3))
\end{lstlisting}

Результат: (-9 -8 A -7)

\item \textbf{Написать функцию, которая возвращает первый аргумент списка-аргумента, который сам является непустым списком.}

На вход функции список. 

\textbf{С использованием функционалов}

Используется функционал reduce, определяется функция found, для поиска списка. Данная функция будет применяться каскадным образом (к первым двум, затем к результату и следующему и так далее). Как только находится первый список, немедленно возвращается это значение без проверки остальных. 

\begin{lstlisting}
(defun first_list (lst)
  (defun found (lst1 lst2)
    (or (if (listp lst1) lst1 Nil)
        (if (listp lst2) lst2 Nil)
    )
  )
  (reduce #'found lst)
)
\end{lstlisting}

\textbf{Примеры:}

\begin{lstlisting}
(first_list '(1 (1) (3)))
\end{lstlisting}

Результат: (1)

\textbf{С использованием рекурсии}

Пока список не Nil (каждый раз функция применяется для хвоста списка), ищется список, если список, то он возвращается. 

\begin{lstlisting}
(defun first_list_rec (lst)
  (cond ((null lst) Nil)
        ((if (listp (car lst)) (car lst)))
        (t (first_list_rec (cdr lst)))
  )
)
\end{lstlisting}

\textbf{Примеры:}

\begin{lstlisting}
(first_list_rec '(1 (2 3 4) (3)))
\end{lstlisting}

Результат: (2 3 4)

\item \textbf{Написать функцию, которая выбирает из заданного списка только те числа, которые между двумя заданными границами. }

На вход принимаются границы диапазона и список. 

\textbf{С использованием функционалов}

Используется функционал reduce, определяется функция found, для проверки вхождения в диапазон. Данная функция будет применяться каскадным образом (к первым двум, затем к результату и следующему и так далее). Следовательно, нужно проверить является ли первый аргумент числом (в первый раз), если так то проверяются оба, нет один. С помощью функции append создается список всех подходящих значений. 

\begin{lstlisting}
(defun found_between (a b lst)
  (defun found (lst1 lst2)
    (if (numberp lst1)
      (append(if(and(< a lst1)(< lst1 b))(list lst1) Nil)
              (if(and(< a lst2)(< lst2 b))(list lst2) Nil))
      (append lst1
              (if(and(< a lst2)(< lst2 b))(list lst2) Nil))
    )
  )
  (reduce #'found lst)
)
\end{lstlisting}

\textbf{Примеры:}

\begin{lstlisting}
(found_between 1 5 '(1 2 3 4 5 6 7 8 9))
\end{lstlisting}

Результат: (2 3 4)

\textbf{С использованием рекурсии}

Пока список не Nil (каждый раз функция применяется для хвоста списка), ищутся все числа в заданном диапазоне. 

\begin{lstlisting}
(defun found_between_rec (a b lst)
  (if lst
    (append
      (if (and (< a (car lst)) (< (car lst) b)) 
      	(list (car lst)) Nil
      )
      (found_between_rec a b (cdr lst))
    )
  )
)
\end{lstlisting}

\textbf{Примеры:}

\begin{lstlisting}
(found_between_rec 1 5 '(1 2 3 4 5 6 7 8 9))
\end{lstlisting}

Результат: (2 3 4)

\item \textbf{Написать функцию, вычисляющую декартово произведение двух своих списков-аргументов. ( Напомним, что А х В это множество всевозможных пар (a b), где а принадлежит А, b принадлежит В.)}

На вход принимаются 2 списка. 

\textbf{С использованием функционалов}

Используется функционалы mapcan, mapcar, $\lambda$-функция (создающая список пробегаясь для каждого Х по всем У) применяется к первым элементам списков, затем ко вторым и т.д., и результаты применения собираются в результирующий список. mapcar отличается от mapcan: mapcan при построении новых данных использует память исходных данных.


\begin{lstlisting}
(defun decart (lst1 lst2)
  (mapcan #'
    (lambda (x)
      (mapcar #'(lambda (y) (list x y)) lst2)
    ) lst1
  )
)
\end{lstlisting}

\textbf{Примеры:}

\begin{lstlisting}
(decart '(1 2 3) '(4 5 6))
\end{lstlisting}

Результат: ((1 4) (1 5) (1 6) (2 4) (2 5) (2 6) (3 4) (3 5) (3 6))

\textbf{С использованием рекурсии}

Пока список не Nil (каждый раз функция применяется для хвоста списка), происходит формирование списка из головы текущего и всех элементов второго списка (для этого используется 2 рекурсивная функция). 

\begin{lstlisting}
(defun decart_rec(x y)
  (cond
    ((null x) nil)
    (t(append(second_param(car x) y)(decart_rec(cdr x) y)))
  )
)

(defun second_param(x y)
  (cond
    ((null y) nil)
    (t (cons (list x (car y)) (second_param x (cdr y))))
  )
)
\end{lstlisting}

\textbf{Примеры:}

\begin{lstlisting}
(decart_rec '(1 2 3) '(4 5 6))
\end{lstlisting}

Результат: ((1 4) (1 5) (1 6) (2 4) (2 5) (2 6) (3 4) (3 5) (3 6))

\item \textbf{Почему так реализовано reduce, в чем причина?}

(reduce $\#$'+ ()) $\to$ 0

Обратимся к исходному коду (точнее его части, где описывается функция reduce). 

\begin{lstlisting}[basicstyle=\footnotesize]
LISPFUN(reduce,seclass_default,2,0,norest,key,5,
        (kw(from_end),kw(start),kw(end),kw(key),kw(initial_value)) )
{ /* (REDUCE function sequence [:from-end] [:start] [:end] [:key]
             [:initial-value]), CLTL p. 251, CLTL2 p. 397
  Stack layout: function, sequence, from-end, start, end, key, initial-value. */
  pushSTACK(get_valid_seq_type(STACK_5)); /* check sequence */
  /* Stack layout: function, sequence, from-end, start, end, key, initial-value,
                  typdescr. */
  check_key_arg(&STACK_(1+1)); /* key check */
  start_default_0(STACK_(3+1)); /* Default value for start is 0 */
  /* Default value for end is the length of the sequence: */
  end_default_len(STACK_(2+1),STACK_(5+1),STACK_0);
  /* check start- and end arguments: */
  test_start_end(&O(kwpair_start),&STACK_(2+1));
  { /* subtract and compare start- and end arguments: */
    var object count = I_I_minus_I(STACK_(2+1),STACK_(3+1));
    /* count = (- end start), an integer >=0. */
    if (eq(count,Fixnum_0)) { /* count = 0 ? */
      /* start and end are equal */
      if (!boundp(STACK_(0+1))) { /* initial-value supplied? */
        /* no -> call function with 0 arguments: */
        funcall(STACK_(6+1),0);
      } else {
        /* yes -> initial-value as result value: */
        VALUES1(STACK_(0+1));
      }
      skipSTACK(7+1);
      return;
    }
    /* common case: start < end, count > 0 */
    pushSTACK(count);
  }
  /* Stack layout: function, sequence, from-end, start, end, key, initial-value,
                   typdescr, count. */
  /* check from-end: */
  ...
\end{lstlisting}

Остальная часть функции не обрабатывается при наших аргументах, а именно зайдет в условие на 18 строчке и на 28 выйдет. При этом присвоится значение 0, так как значение по умолчанию для суммы (вызов + для 0 аргументов на строчке 22), при заданном начальном значении присвоится оно. 

Например:

(reduce $\#$'+ () :initial-value 1) $\to$ 1

(reduce $\#$'* ()) $\to$ 1

\end{enumerate}

\textbf{Теоретические вопросы:}

\begin{enumerate}
\item Способы организации повторных вычислений в Lisp. 

Использование функционалов или рекурсии. 

\item Различные способы использования функционалов. 

(mapcar/maplist $\#$'func lst)

mapcar -- функция func применяется к головам первым элементам списков, затем ко вторым и т.д., и результаты применения собираются в результирующий список.

maplist -- func применяется к “хвостам” списков, начиная с полного списка.

mapcan, mapcon -- аналогичны mapcar и maplist, используется память исходных данных, не работают с копиями.

(reduce $\#$’func lst)

reduce -- функция func применяется каскадным образом (к первым двум, затем к результату и следующему и так далее).

\item Что такое рекурсия? Способы организации рекурсивных функций. 

Рекурсия -- это ссылка на определяемый объект во время его определения. 

\begin{enumerate}
\item Хвостовая рекурсия -- результат формируется не на выходе из рекурсии, а на входе в рекурсию, все действия выполняя до ухода на следующий шаг рекурсии. 
\item Рекурсия по нескольким параметрам
\item Дополняемая рекурсия -- при обращении к рекурсивной функции используется дополнительная функция не в аргументе вызова, а вне его
\item Множественная рекурсия -- на одной ветке происходит сразу несколько рекурсивных вызовов.
\end{enumerate}

\item Способы повышения эффективности реализации рекурсии. 

Использование хвостовой рекурсии. Если условий выхода несколько, то надо думать о порядке их следования. Некачественный выход из рекурсии может привести к переполнению памяти из-за "лишних" рекурсивных вызовов. 

Преобразование не хвостовой рекурсии в хвостовую, возможно путем использования дополнительных параметров. В этом случае необходимо использовать функцию-оболочку для запуска рекурсивной функции с начальными значениями дополнительных параметров. 

\end{enumerate}
\end{document}