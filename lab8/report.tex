\documentclass[a4paper,14pt]{extreport} % формат документа

\usepackage{amsmath}
\usepackage{cmap} % поиск в ПДФ
\usepackage[T2A]{fontenc} % кодировка
\usepackage[utf8]{inputenc} % кодировка исходного текста
\usepackage[english,russian]{babel} % локализация и переносы
\usepackage[left = 2cm, right = 1cm, top = 2cm, bottom = 2 cm]{geometry} % поля
\usepackage{listings}
\usepackage{graphicx} % для вставки рисунков
\usepackage{amsmath}
\usepackage{float}
\usepackage{multirow}
\graphicspath{{pictures/}}
\DeclareGraphicsExtensions{.pdf,.png,.jpg}
\newcommand{\anonsection}[1]{\section*{#1}\addcontentsline{toc}{section}{#1}}

\lstset{ %
	language=Lisp,                % Язык программирования 
	numbers=left,                   % С какой стороны нумеровать          
	frame=single,                    % Добавить рамку
}

\begin{document}
\begin{titlepage}

    \begin{table}[H]
        \centering
        \footnotesize
        \begin{tabular}{cc}
            \multirow{8}{*}{\includegraphics[scale=0.35]{bmstu.jpg}}
            & \\
            & \\
            & \textbf{Министерство науки и высшего образования Российской Федерации} \\
            & \textbf{Федеральное государственное бюджетное образовательное учреждение} \\
            & \textbf{высшего образования} \\
            & \textbf{<<Московский государственный технический} \\
            & \textbf{университет имени Н.Э. Баумана>>} \\
            & \textbf{(МГТУ им. Н.Э. Баумана)} \\
        \end{tabular}
    \end{table}

    \vspace{-2.5cm}

    \begin{flushleft}
        \rule[-1cm]{\textwidth}{3pt}
        \rule{\textwidth}{1pt}
    \end{flushleft}

    \begin{flushleft}
        \small
        ФАКУЛЬТЕТ
        \underline{<<Информатика и системы управления>>\ \ \ \ \ \ \ 
        \ \ \ \ \ \ \ \ \ \ \ \ \ \ \ \ \ \ \ \ \ \ \ \ \ \ \ \ \ \ \ 
    \ \ \ \ \ \ \ \ \ \ \ \ \ \ \ } \\
        КАФЕДРА
        \underline{<<Программное обеспечение ЭВМ и
        информационные технологии>>
        \ \ \ \ \ \ \ \ \ \ \ \ \ \ \ \ \ \ \ \ }
    \end{flushleft}

    \vspace{2cm}

    \begin{center}
        \textbf{Лабораторная работа № 8} \\
        \vspace{0.5cm}
        \textbf{} \\
    \end{center}

    \vspace{4cm}

    \begin{flushleft}
        \begin{tabular}{ll}
            \textbf{Дисциплина} & Функциональное и логическое программирование \\
            \textbf{Студент} & Сиденко А.Г. \\
            \textbf{Группа} & ИУ7-63Б \\
            \textbf{Преподаватель} & Толпинская Н.Б., Строгано Ю.В.  \\
        \end{tabular}
    \end{flushleft}

    \vspace{4cm}

   \begin{center}
        Москва, 2020 г.
    \end{center}

\end{titlepage}

\begin{enumerate}
\item \textbf{Написать функцию, которая по своему списку-аргументу Ist определяет является ли он палиндромом (то есть равны ли Ist и (reverse Ist)).}

\begin{lstlisting}
(defun palindrom (lst)
  (equalp lst (reverse lst)) 
)
\end{lstlisting}

\item \textbf{Напишите функцию swap-first-last, которая переставляет в списке-аргументе первый и последний элементы.}

На вход подается список. 

Функция last$\_$elem ищет последний элемент в списке. 

Функция centr возвращает список без последнего элемента. 

Функция swap$\_$first$\_$last объединяет последний, середину и первый элемент исходного списка. 
Символ ` блокирует вычисления, но эту блокировку можно прервать с помощью другого символа , .

\begin{lstlisting}
(defun last_elem (lst)
  (if (NULL (cadr lst)) 
    (car lst) (last_elem (cdr lst))
  ) 
)

(defun centr (lst)
  (if (NULL (cadr lst)) 
    ()
    (cons (car lst)(centr (cdr lst)))
  )
)

(defun swap_first_last (lst)
  `(,(last_elem lst) ,@(centr (cdr lst)) ,(car lst)) 
)
\end{lstlisting}

\item \textbf{Напишите функцию swap-two-ellement, которая переставляет в списке-аргументе два указанных своими порядковыми номерами элемента в этом списке.}

На вход подается список и 2 числа. 

Используется та же идея объединения списков, что и в предыдущем пункте, но сначала проверяем чтобы первый порядковый номер был меньше второго, в противном случае вызывается функция с переставленными аргументами. 

Объединяются списки с 0 до i, j элемент, c i+1 до j, i элемент и c j+1 до конца. 

\begin{lstlisting}
(defun swap_two_ellement (lst i j)
  (cond ((= i j) lst)
    ((> i j) (swap_two_ellement lst j i))
    (t
     `(,@(subseq lst 0 i),(nth j lst),@(subseq lst(+ i 1)j)
      ,(nth i lst) ,@(subseq lst (+ j 1))))
  ) 
)
\end{lstlisting}

\item \textbf{Напишите две функции, swap-to-left и swap-to-right, которые производят круговую перестановку в списке-аргументе влево и вправо, соответственно.}

На вход подается список и число. 

Та же идея объединения списков, в зависимости от перестановки влево или вправо, объединяются разные списки. 
Если количество позиций в перестановке больше длины, берется остаток от деления функция rem. 

\begin{lstlisting}
(defun swap_to_left (lst k)
  (cond ((= k 0) lst)
        ((> k (length lst)) 
         (swap_to_left lst (rem k (length lst))))
        (t
         `(,@(subseq lst k) ,@(subseq lst 0 k)))
   ) 
)

(defun swap_to_right (lst k)
 (cond ((= k 0) lst)
       ((> k (length lst)) 
        (swap_to_right lst (rem k (length lst))))
       (t
        `(,@(subseq lst (- (length lst) k)) 
        ,@(subseq lst 0 (- (length lst) k))))
 ) 
)
\end{lstlisting}

\item \textbf{Напишите функцию, которая умножает на заданное число-аргумент все числа
из заданного списка-аргумента, когда все элементы списка --- числа или элементы списка -- любые объекты.}

На вход подается список и число. 

С помощью функционала mapcar, производится работа с каждым аргументом списка: проверяется на число (если необходимо), выполняется умножение на заданный аргумент. 

\begin{lstlisting}
(defun multiplication_numbers (lst k)
  (mapcar #'(lambda (x) (* x k)) lst)
)

(defun multiplication_all (lst k)
  (mapcar #'(lambda (x) (if (numberp x) (* x k) x)) lst)
)
\end{lstlisting}

\item \textbf{Напишите функцию, select-between, которая из списка-аргумента,
содержащего только числа, выбирает только те, которые расположены между двумя указанными границами-аргументами и возвращает их в виде списка (упорядоченного по возрастанию списка чисел).}

На вход подается 2 числа и список. 

Используется функционал reduce, определяется функция found, для проверки вхождения в диапазон. Данная функция будет применяться каскадным образом (к первым двум, затем к результату и следующему и так далее). Следовательно, нужно проверить является ли первый аргумент числом (в первый раз), если так то проверяются оба, нет один. С помощью функции append создается список всех подходящих значений. 

\begin{lstlisting}
(defun found_between (a b lst)
  (defun found (lst1 lst2)
    (if (numberp lst1)
      (append(if(and(< a lst1)(< lst1 b))(list lst1) Nil)
              (if(and(< a lst2)(< lst2 b))(list lst2) Nil))
      (append lst1
              (if(and(< a lst2)(< lst2 b))(list lst2) Nil))
    )
  )
  (reduce #'found lst)
)
\end{lstlisting}

\end{enumerate}
\end{document}